\documentclass[aspectratio=169]{beamer}

\usetheme{Madrid}

\title{Python Crash Course}
\author{Králik Balázs}
\date{\today}

\begin{document}

\begin{frame}
    \titlepage
\end{frame}

\begin{frame}{Outline}
    \tableofcontents
\end{frame}

\section{Introduction}
\begin{frame}{Introduction}
\begin{itemize}
    \item Welcome to the Python Crash Course
    \item Focus on statistical analysis with real data
    \item Learn Python basics through practical examples
    \item Cover pandas, matplotlib, and statsmodels
    \item Work with panel data and regression analysis
\end{itemize}    
\end{frame}

\section{Downloads}
\begin{frame}{Downloads}
\begin{itemize}
    \item Download Python 3.x from \url{https://www.python.org/downloads/}. Click "put on path."
    \item Download Visual Studio Code from \url{https://code.visualstudio.com/download}
    \item Download Git from \url{https://git-scm.com/downloads}. Make sure to click "Git Bash" option
    \item Create a GitHub account at \url{https://github.com}.
    \item For Latex
    \begin{itemize}
        \item Download MikTex from \url{https://miktex.org/download}
        \item Download Strawberry perl from \url{https://strawberryperl.com/}
    \end{itemize}
\end{itemize}
\end{frame}

\section{Setup}
\begin{frame}{Setup}
\begin{itemize}
    \item Open Visual Studio Code and Git clone the project from 
          \url{https://github.com/balazs14/python-course.git}.
    \item New terminal, initialize venv as per README
    \item Click python\_intro\_stats\_notebook.ipynb
    \item Install suggested extensions 
    \item Install Latex Workshop for latex.
\end{itemize}
\end{frame}

\section{Sample Analysis}
\begin{frame}{Sample Analysis Results}
\begin{figure}[h]
    \centering
    \includegraphics[width=0.4\textwidth]{regression_plot.png}
    \caption{Some Fancy Analysis}
    \label{fig:sample_analysis}
\end{figure}
As we can see in Figure \ref{fig:sample_analysis}, our publication deserves a lot of citations.
\end{frame}

\begin{frame}
    \begin{center}
        \Huge Questions?
    \end{center}
\end{frame}

\end{document}